
\hspace*{\parindent}%
Системы массового обслуживания с орбитами, также называемые RQ-системами, обладают большой популярностью, иначе книги посвящённые только данной теме, количество страниц у которых исчесляются в сотнях,  не существовали бы [11, 12]. Но в нашем иследовании намного больше пригодилась статья [1]. В ней использовался метод ассимптотически диффузионного анализа, использующее асимптотическое условие предельное малой интенсивности заявки на орбите.

Если в первой главе хватает одномерного процесса для отслеживания состояния блока обслуживания, то во второй главе, пришлось последовать примеру статьи[3] и вводить двумерный процесс, что в конечном итоге переростает в трёхмерный, так как нам нужно следить не только за состоянием блока обслуживания, но и за состоянием орбиты.

На самом деле исследование двухфазных систем в условиях большой задержки на орбите уже проводилось, даже при участии одного из авторов данной статьи - Назарова Анатолия Андреевича [13]. Но две фазы там были расположены последовательно, а не как в данной работе выбиралась одна из двух с определённой вероятностью, с орбиты заявка могла поступить только на вторую фазу, обратной связи не было, да и использован был в статье[13] метод симптотического анализа, а не асимптотически диффузионного. Но также и проводилось исследования использующие асимптотически диффузионный анализ, но в данной статье у приборов отсутствует обратная связь.

Также помогли книги[2,4,5,6,7,8,10] для ознакомления с различными методами.

\textbf{Цель дипломной работы:} построить ряд распределения, или его апроксимацию, для количества заявок на орите для RQ-системы M|$H_2$|N в стационарном режиме.

\textbf{Задачи:}\\
1.Построить математическую модель систем $M|H_{2}|2$, $M|H_{2}|N$ с обратной связью.\\
2.Составить систему дифференциальных уравнений Колмогорова для систем $M|H_{2}|2$, $M|H_{2}|N$ с обратной связью.\\
3.С помощью метода асимптотического анализа найти коэффициенты переноса и диффузии дифференциальных уравнений систем $M|H_{2}|2$, $M|H_{2}|N$ с обратной связью.\\
4.С помощью метода асимптотически диффузионного анализа вычислить плотность распределений вероятностей произвольного количества заявок на орбите и получить дискретные распределения вероятностей.

 
