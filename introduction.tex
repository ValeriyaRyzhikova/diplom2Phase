\hspace*{\parindent}%
Системы массового обслуживания с орбитами, также называемые RQ-системы, обладают большой популярностью и широко рассмотрены в литературе [6, 8, 12, 13]. 

Очень похожая работа была также рассмотрена в [20]. В ней, так же как и в этой работе, гиперэкспаненциальное время обслуживания и, если заявка пришла в тот момент, когда все приборы заняты, то она так же отправляется на орбиту, где ожидает время, распределённое по экспоненциальному закону. Однако, после завершения обслуживания, в работе [20] заявка покидает в систему, в то время как в данной работе заявка может также уйти на орбиту или же мгновенно перейти на повторное обслуживание. И так же, с помощью асимптотически диффузионного анализа был найден ряд распределения количества заявок на орбите.


В данной работе рассматриваются двухфазные системы $M|H_{2}|2$, $M|H_{2}|N$ с обратной связью.

Исследование двухфазных систем проводилось [7]. Но принципиальное отличие предложенной системы состоит в том, что в системе, исследуемой в [7] фазы расположены последовательно, с орбиты заявка поступает только на вторую фазу, обратной связи нет. А в данной работе модель предполагает, что заявка из входящего потока выбирает одну из двух фаз обслуживания с определённой вероятностью и имеется обратная связь.

В первой главе исследуется система $M|H_{2}|2$ методом асимптотического анализа в асимптотическом условии предельно малой интенсивности обращений заявок с орбиты. В стационарном режиме получено распределение вероятностей числа занятых приборов на первой и второй фазе, а также построена аппроксимация ряда распределения вероятностей числа заявок на орбите. Приведены численные примеры.
В первой главе для получения результата был рассмотрен одномерный процесс, характеризующий состояние блока обслуживания, однако во второй главе для исследования системы $M|H_{2}|N$ рассматривается двумерный процесс, характеризующий состояние блока обслуживания и орбиты. Для исследования применялся метод асимптотически диффузионного анализа, использующее асимптотическое условие предельное малой интенсивности заявки на орбите [1].
Для системы $M|H_{2}|N$ стационарном режиме найдено распределение вероятностей числа занятых приборов на первой и второй фазах, а также построена аппроксимация ряда распределения вероятностей числа заявок на орбите в стационарном режиме. Приведены результаты численных экспериментов.


Но в нашем исследовании намного больше пригодилась статья [21].

Также помогли книги [1,2,3,5,9,10,11] для ознакомления с различными методами.

\textbf{Цель дипломной работы:} построить ряд распределения, или его апроксимацию, для количества заявок на орите для RQ-системы $M|H_2|N$ в стационарном режиме.

\textbf{Задачи:}\\
1. Построить математическую модель систем $M|H_{2}|2$, $M|H_{2}|N$ с обратной связью.\\
2. Составить систему дифференциальных уравнений Колмогорова для систем $M|H_{2}|2$, $M|H_{2}|N$ с обратной связью.\\
3. С помощью метода асимптотического анализа найти коэффициенты переноса и диффузии дифференциальных уравнений систем $M|H_{2}|2$, $M|H_{2}|N$ с обратной связью.\\
4. С помощью метода асимптотически диффузионного анализа вычислить плотность распределений вероятностей произвольного количества заявок на орбите и получить дискретные распределения вероятностей.
