
\hspace*{\parindent}%
Передача данных это большая проблема, особенно сейчас, когда передача данных очень часто используется для предоставления интернет или телекоммуникационных услуг. При исследовании данных проблем часто используются системы массового обслуживания [4, 5, 7, 11], в том числе двухфазные системы [3, 14, 15], а также когда имеется большое количество абонентов, системы с орбитой, также называемы RQ-системами [1, 10, 12, 13]. Данная работа же отличается совмещения двух этих особенностей с тем фактом, что система N-линейна, или говоря другими словами имеет N приборов. 

\textbf{Цель дипломной работы:} исследовать N-линейную систему M|$H_{2}$|N с обратной связью.\\

\textbf{Задачи:}\\
1.Построить математическую модель N-линейной системы M|$H_{2}$|M с обратной связью.\\
2.Составить систему дифференциальных уравнений Колмогорова.\\
3.С помощью метода асимптотического анализа найти коэффициенты переноса и диффузии дифференциального уравнения N-линейной системы M|$H_{2}$|N с обратной связью.\\
4.С помощью метода асимптотически диффузионного анализа вычислить плотность распределения вероятностей произвольного числа заявок на орбите и получить дискретное распределение вероятностей.

 
