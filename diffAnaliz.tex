Тогда \eqref{withB} перепишем в виде
\begin{align}\label{newWithB}
	\frac{\partial \Phi (w,\tau)}{\partial \tau}=a'(x) w\frac{\partial \Phi (w,\tau)}{\partial w}+\frac{(jw)^2}{2}b(x)\Phi(w,\tau)
\end{align}
Уравнение c это преобразование Фурье уравнения Фокера-Планка для плотности распределения вероятностей $P(y, \tau )$ значений центрированного и нормированного количества заявок на орбите. Находя обратное преобразование Фурье от \eqref{newWithB}, получим
\begin{align}
	\frac{\partial P (y,\tau)}{\partial \tau}=-\frac{\partial}{\partial y}\{a'(x)yP(y,\tau)\} 
	+\frac{1}{2}\frac{\partial^2}{\partial y^2}\{b(x)P(y,\tau)\}.
\end{align}
Следовательно $P (y,\tau)$ плотность распределения вероятностей диффузионного процесса [10], который обозначим $y(\tau)$ с коэффициентом переносом $a(x)$ и коэффициентом диффузии $b(x)$
\begin{align}
	dy(\tau)=a'(x)yd\tau+\sqrt{b(x)}dw(\tau).
\end{align}
Рассмотрим стохастический процесс нормированного числа заявок на орбите
\begin{align}
	z(\tau)=x(\tau)+\varepsilon y(\tau),
\end{align}
где $\varepsilon=\sqrt{\sigma}$, исходя из  \eqref{a_func}, $dx(\tau)=a(x)d\tau$, следует
\begin{align}\label{bz}
	dz(\tau)=d(x(\tau)+\varepsilon y(\tau))=(a(x)+\varepsilon ya'(x))d\tau+\varepsilon \sqrt{b(x)}dw(\tau).
\end{align}
Разложим $a(z)$ в ряд 
\begin{align*}
	&a(z)=a(x+\varepsilon y)=a(x)+\varepsilon y a'(x)+O(\varepsilon^2),\\
	&\varepsilon\sqrt{b(z)}=\varepsilon\sqrt{b(x+\varepsilon y)}=\varepsilon\sqrt{b(x)+O(\varepsilon)}=\sqrt{\sigma b(x)}+O(\varepsilon).
\end{align*}
Перепишем уравнение \eqref{bz} с точностью до $O(\varepsilon^2)$
\begin{align}\label{newBz}
	dz(\tau)=a(z)d\tau+\sqrt{\sigma b(z)}dw(\tau).
\end{align}
Обозначим плотность распределения вероятностей для процесса $z(\tau)$
\begin{align*}
	\pi(z,\tau)=\frac{\partial P\{z(\tau)<z\}}{\partial z}.
\end{align*}
Так как $z(\tau)$ -- это решение стохастического дифференциального уравнения \eqref{newBz}, следовательно, процесс является диффузионным и для его плотности распределения вероятностей можем записать уравнение Фокера-Планка
\begin{align}
	\frac{\partial \pi (z,\tau)}{\partial \tau}=-\frac{\partial}{\partial z}\{a(z)\pi(z,\tau)\} 
	+\frac{1}{2}\frac{\partial^2}{\partial z^2}\{\sigma b(z)\pi(z,\tau)\}.
\end{align}
Предполагая, что существует стационарный режим, обозначим 
\begin{align}
	\pi (z,\tau)=\pi(z),
\end{align}
запишем уравнение Фокера-Планка для стационарного распределения вероятностей $\pi{(z)}$
\begin{align*}
	(a(z)\pi(z))'+\frac{\sigma}{2}(b(z)\pi(z))''=0,\\
	-a(z)\pi(z)+\frac{\sigma}{2}(b(z)\pi(z))'=0.
\end{align*}
Решая данную систему уравнений получаем плотность распределения вероятностей $\pi{(z)}$ нормированного числа заявок на орбите
\begin{align}
	\pi (z)= \frac{C}{b(z)}exp\bigg\{\frac{2}{\sigma} \int\limits_0^z \frac{a(x)}{b(x)}dx\bigg\}.
\end{align} 
Теорема доказана.

Получим дискретное распределение вероятностей
\begin{align}
	P(i)=\pi(\sigma i)/\sum\limits_{i=0}^{\infty} \pi(\sigma i),
\end{align} 
которое будем называть диффузионной аппроксимацией дискретного распределения вероятностей количества заявок на орбите для изучаемой системы.


Нетрудно показать, что условием существования стационарного
режима рассматриваемой системы является неравенство 
\begin{align}\label{uslStat}
	\lambda<Nr_{0}\ (\frac{q}{\mu_{1}}+\frac{1-q}{\mu_{2}}).
\end{align}

Введем следующую замену для того, чтобы среднее время обслуживания равнялось единице
\begin{align*}
	q=\frac{\mu_{1}(1-\mu_{2})}{\mu_{1}-\mu_{2}}.
\end{align*}
В таком случае неравенство \eqref{uslStat} имеет вид
\begin{align*}
	\lambda<Nr_{0}.
\end{align*}