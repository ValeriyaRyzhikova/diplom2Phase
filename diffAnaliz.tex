Сравнивая это равенство и \eqref{parthWithA}, запишем уравнение \eqref{withB} в виде
\begin{align}\label{newWithB}
\frac{\partial \Phi (w,\tau)}{\partial \tau}=a'(x) w\frac{\partial \Phi (w,\tau)}{\partial w}+\frac{(jw)^2}{2}b(x)\Phi(w,\tau)
\end{align}
Уравнение \eqref{newWithB} это преобразование Фурье уравнения Фокера-Планка для плотности распределения вероятностей $P(y, \tau )$ значений центрированного и номированного количества заявок на орбите. Находя обратное преобразование Фурье от \eqref{newWithB} мы получаем
\begin{align}
	\frac{\partial P (y,\tau)}{\partial \tau}=-\frac{\partial}{\partial y}\{a'(x)yP(y,\tau)\} 
	+\frac{1}{2}\frac{\partial^2}{\partial y^2}\{b(x)P(y,\tau)\}
\end{align}
Если мы составили уравнение Фокера-Планка для функции, значит это функция является плотностью распределения вероятностей для диффузионного процесса [2], который мы обозначим $y(\tau)$ с коэффецентом переносом a(x) и коэффецентом диффузии b(x) 
\begin{align}
	dy(\tau)=a'(x)yd\tau+\sqrt{b(x)}dw(\tau).
\end{align}
Рассматривая стохастический процесс нормированного числа заявок на орбите
\begin{align}
	z(\tau)=x(\tau)+\varepsilon y(\tau),
\end{align}
где $\varepsilon=\sqrt{\sigma}$, исходя из \eqref{a(x)} $dx(\tau)=a(x)d\tau$, следовательно
\begin{align}\label{bz}
	dz(\tau)=d(x(\tau)+\varepsilon y(\tau))=(a(x)+\varepsilon ya'(x))d\tau+\varepsilon \sqrt{b(x)}dw(\tau).
\end{align}
Затем выполняя разложение получаем 
\begin{align*}
	&a(z)=a(x+\varepsilon y)=a(x)+\varepsilon y a'(x)+O(\varepsilon^2),\\
	&\varepsilon\sqrt{b(z)}=\varepsilon\sqrt{b(x+\varepsilon y)}=\varepsilon\sqrt{b(x)+O(\varepsilon)}=\sqrt{\sigma b(x)}+O(\varepsilon)
\end{align*}
После переписываем уравнение \eqref{bz} с точностью до $O(\varepsilon^2)$
\begin{align}\label{newBz}
	dz(\tau)=a(z)d\tau+\sqrt{\sigma b(z)}dw(\tau).
\end{align}
Обозначим плотность распределения вероятностей для процесса $z(\tau)$
\begin{align*}
	\pi(z,\tau)=\frac{\partial P\{z(\tau)<z\}}{\partial z}
\end{align*}
Так как $z(\tau)$ это решение стохастического дифференциального уравнения \eqref{newBz}, следовательно процесс является диффузионным процессом и для его плотности распределения вероятностей мы можем записать уравнение Фокера-Планка
\begin{align}
\frac{\partial \pi (z,\tau)}{\partial \tau}=-\frac{\partial}{\partial z}\{a(z)\pi(z,\tau)\} 
+\frac{1}{2}\frac{\partial^2}{\partial z^2}\{\sigma b(z)\pi(z,\tau)\}.
\end{align}
Предполагая, что существует стационарный режим 
\begin{align}
\pi (z,\tau)=\pi(z),
\end{align}
Пишем уравнение Фокера-Планка для стационарного распределения вероятностей $\pi{(z)}$ 
\begin{align*}
(a(z)\pi(z))'+\frac{\sigma}{2}(b(z)\pi(z))''=0,\\
-a(z)\pi(z)+\frac{\sigma}{2}(b(z)\pi(z))'=0.
\end{align*}
Решая данную систему уравнений мы получаем плотность распределения вероятностей $\pi{(z)}$ нормировонного числа заявок на орбите
\begin{align}
\pi (z)= \frac{C}{b(z)}exp\{\frac{2}{\sigma} \int\limits_0^z \frac{a(x)}{b(x)}dx\}
\end{align} 
После чего можем получить дискретное распределение вероятностей
\begin{align}
P(i)=\pi(\sigma i)/\sum\limits_{i=0}^{\infty} \pi(\sigma i)
\end{align} 
Которое называем диффузионной аппроксимацией дискретного распределения вероятностей заявок на орбите для изучаемой системы.\\

Нетрудно показать, что условие существования стационарного
режима рассматриваемой системы является неравенство[8]
\begin{align}\label{uslStat}
\lambda<Nr_{0}\ (\frac{q}{\mu_{1}}+\frac{1-q}{\mu_{2}}).
\end{align}

Введем следующую замену для того, чтобы среднее время обслуживания равнялось еденице.
\begin{align*}
q=\frac{\mu_{1}(1-\mu_{2})}{\mu_{1}-\mu_{2}}
\end{align*}
В таком случаее неравество \eqref{uslStat} имеет вид
\begin{align*}
\lambda<Nr_{0}.
\end{align*}