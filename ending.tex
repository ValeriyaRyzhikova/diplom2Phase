В настоящей работе представлено исследование RQ-систем с обратной связью вида $M|H_{2}|2$, $M|H_{2}|N$. Анализ моделей, выполненный в работе, представлен двумя методами исследования: методом асимптотического и асимптотически диффузионного анализа. В работе построены аппроксимации распределений вероятностей числа приборов, занятых на первой и второй фазе в указанных системах в условии большой задержки заявок на орбите. Также построены диффузионные аппроксимации распределений вероятностей числа заявок на орбите для всех систем. Приведено несколько численных примеров и сравнения с результатами имитационного моделирования. 

Основными научными достижениями данного исследования являются:
\begin{enumerate} 
	\item Нахождение аппроксимации распределений вероятностей числа приборов, занятых на первой и второй фазе в системах $M|H_{2}|2$, $M|H_{2}|N$. 
	\item Модификации метода асимптотического анализа для исследования RQ-систем с обратной связью в предельных условиях согласованно высокой интенсивности вызывания заявок, согласованно длительного обслуживания вызываемых заявок и большой задержки заявок на орбите.
	\item Алгоритм применения метода асимптотически-диффузионного анализа для исследования RQ-систем с обратной связью.
	\item Аппроксимации распределений вероятностей числа заявок на орбите в RQ-системах с обратной связью при различных условиях функционирования систем.
\end{enumerate}

Результаты, описанные в данной работе, были представлены в докладах на следующих конференциях:
\begin{enumerate} 
	\item VII Международная молодежная научная конференция <<Математическое и программное обеспечение информационных, технических и экономических систем>>, г. Томск, 28--30 мая, 2020;
	\item VIII Международная молодежная научная конференция <<Математическое и программное обеспечение информационных, технических и экономических систем>>, г. Томск, 22--30 мая, 2021.
\end{enumerate}.
