Настоящая работа изучает систему массового обслуживания, с простейшим потоком, орбитой, а также произвольным количеством приборов с гиперэкспоненциальным распределением и обратной связью.

\textbf{Ключевые слова:}теория массового обслуживания, система массового обслуживания, RQ-система, характеристическая функция, метод асимптотического анализа, метод асимптотически диффузионного анализа, простейший поток, гиперэкспоненциальное распределение, произвольное количество приборов, орбита.

\textbf{Объект исследования:} RQ-система M|$H_2$|N.

\textbf{Цель:} построить ряд распределения, или его апроксимацию, для количества заявок на орите для RQ-системы M|$H_2$|N в стационарном режиме.

\textbf{Структура работы:} настоящая работа включает в себя 2 раздела, 46 страниц, 8 рисунка, 20 источников литературы.

Начиная с малого в первой главе мы получаем путем асимптотически-диффузионного анализа, использую асимптотическое условие предельно малой интенсивности заявки на орбите, через матричные уравнения распределение количества занятых приборов на первой и второй фазе в стационарном режиме для двух приборов, также как аппроксимацию ряда распределения количества заявок на орбите в стационарном режиме для двух приборов, а после получаем визуальное представление второго результата при конкретных значениях. 
Во второй главе аналогичными действиями, только используя скалярные выражения, вместо матричных выводим распределение количества занятых приборов на первой и второй фазе в стационарном режиме для произвольного положительного целого количества приборов, также как аппроксимацию ряда распределения количества заявок на орбите в стационарном режиме для произвольного положительного целого количества приборов, в итоге построив графики для различного количества приборов.






