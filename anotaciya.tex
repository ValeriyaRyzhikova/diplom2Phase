Настоящая работа посвящена исследованию системы массового обслуживания с простейшим потоком, орбитой, а также произвольным количеством приборов, гиперэкспоненциальным распределением времени обслуживания и с обратной связью.

\textbf{Ключевые слова:} теория массового обслуживания, система массового обслуживания, RQ-система, характеристическая функция, метод асимптотического анализа, метод асимптотически диффузионного анализа, простейший поток, гиперэкспоненциальное распределение, произвольное количество приборов, орбита.

\textbf{Объект исследования:} RQ-система $M|H_2|N$.

\textbf{Цель:} построить ряд распределения вероятностей, или его аппроксимацию, количества заявок на орбите для RQ-системы $M|H_2|N$ в стационарном режиме.

\textbf{Структура работы:} настоящая работа включает в себя 2 раздела, 52 страницы, 8 рисунков, 20 источников литературы.

Начиная с малого, в первой главе путем асимптотически диффузионного анализа, используя асимптотическое условие предельно малой интенсивности обращений заявок с орбиты, получаем распределение вероятностей количества занятых приборов на первой и второй фазе в стационарном режиме для двух приборов, а также аппроксимацию ряда распределения вероятностей количества заявок на орбите в стационарном режиме для двух приборов. После чего получаем визуальное представление второго результата при конкретных значениях. 

Во второй главе аналогичными действиями, только используя скалярные выражения, вместо матричных выводим распределение вероятностей количества занятых приборов на первой и второй фазах в стационарном режиме для произвольного положительного целого количества приборов, а также аппроксимацию ряда распределения вероятностей количества заявок на орбите в стационарном режиме для произвольного положительного целого количества приборов, в итоге, строим графики полученных аппроксимаций распределений вероятностей для различного количества приборов.
