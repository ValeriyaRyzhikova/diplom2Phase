\textbf{Теорема 2.1.} Компоненты  $R_{n_{1}, n_{2}}(x)$ распределения вероятностей числа приборов, занятых на первой и второй фазе имеет вид

\begin{equation}\label{R}
		R_{n_{1}, n_{2}}(x)=\frac{L_{n_{1}, n_{2}}(x)}{c(x)},
\end{equation}
		где\\
\begin{equation*}
	\begin{aligned}
		L_{n_{1}, n_{2}}(x) =& (\mu_{1}\mu_{2}(1-r_{1}))^{N-(n_{1}+n_{2})}
		\frac{N!}{(n_{1}+n_{2})!}C_{n_{1}+n_{2}}^{n_{2}}
		(\mu_{1}(1-q))^{n_{2}}(\mu_{2}q)^{n_{1}}(\lambda+x)^{n_{1}+n_{2}},\\
		c(x)=&\sum_{n_1=0}^N\sum_{n_2=0}^{N-n_1} L_{n_{1}, n_{2}}.
	\end{aligned}
\end{equation*}

\begin{equation*}
	\begin{aligned}
		x=x(\tau);x'(\tau)=a(x)=&\lambda \sum_{n_1=0}^N\sum_{n_2=N-n_1}^{N-n_1} 
		R_{n_{1}, n_{2}}
		- x\sum_{n_1=0}^{N-1}\sum_{n_2=0}^{N-1-n_1} 
		R_{n_{1}, n_{2}}+\\
		&+\mu_1 r_2 \sum_{n_1=0}^{N}\sum_{n_2=0}^{N-n_1} 
		n_1 R_{n_{1}, n_{2}}
		+\mu_2 r_2 \sum_{n_1=0}^{N}\sum_{n_2=0}^{N-n_1} 
		n_2 R_{n_{1}, n_{2}}.
	\end{aligned}
\end{equation*}