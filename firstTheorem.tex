\textbf{Теорема 1.} В рассматриваемой системе с обратной связью компоненты $R_{n_{1}, n_{2}}$ распределения вероятностей определяется уравнением
\begin{equation}
\begin{split} \label{R}
&L_{n_{1}, n_{2}} = (\mu_{1}\mu_{2}(1-r_{1}))^{N-(n_{1}+n_{2})}
\frac{N!}{(n_{1}+n_{2})!}C_{n_{2}}^{n_{1}+n_{2}}
(\mu_{1}(1-q))^{n_{1}}(\mu_{2}q)^{n_{2}}(\lambda+x)^{n_{1}+n_{2}},\\
&c=\sum_{n_1=0}^N\sum_{n_2=0}^{N-n_1} L_{n_{1}, n_{2}},\\
&R_{n_{1}, n_{2}}=\frac{L_{n_{1}, n_{2}}}{c}
\end{split}
\end{equation}

$x=x(\tau):x'(\tau)=a(x)=\lambda \sum_{n_1=0}^N\sum_{n_2=N-n_1}^{N-n_1} 
R_{n_{1}, n_{2}}
- x(\tau)\sum_{n_1=0}^{N-1}\sum_{n_2=0}^{N-1-n_1} 
R_{n_{1}, n_{2}}+\mu_1 r_2 \sum_{n_1=0}^{N-1}\sum_{n_2=0}^{N-1-n_1} 
n_1 R_{n_{1}, n_{2}}
+\mu_2 r_2 \sum_{n_1=0}^{N-1}\sum_{n_2=0}^{N-1-n_1} 
n_2 R_{n_{1}, n_{2}}$\\
