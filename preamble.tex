\documentclass[a4paper,12pt]{report}
\usepackage{subfiles}

\usepackage[utf8]{inputenc}															%входная кодировка - unicode
\usepackage[T2A]{fontenc}															%выходная кодировка - t2a
\usepackage[main=russian]{babel}													%поддержка языков

\usepackage{amsmath, amsfonts, amssymb, textcomp}									%для формул
\usepackage{graphicx}																%для картиночек
\usepackage{caption}																%подписи таблиц и картинок
\DeclareCaptionLabelSeparator{emdash}{~---~}										%разделитель длинное тире
\captionsetup{margin=10pt,labelsep=emdash}											%задание формата подписи

\usepackage[left=3cm,right=1cm,top=2cm,bottom=2cm,bindingoffset=0cm]{geometry}		%поля, стоят как надо

\usepackage{indentfirst} 															%начало первого абзаца с отступа
\setlength{\parindent}{1.25cm}

%\usepackage{mathptm}																%прямое начертание греческих букв, оч всрато

\usepackage{setspace}																%редактировать междустрочный интервал
\onehalfspacing																		%задает полуторный интервал

\usepackage{pscyr}																	%кириллические шрифты
\renewcommand{\rmdefault}{ftm}														%times new roman кириллица

\usepackage{titlesec}																%корректировать заголовки
\titleformat{\chapter}{\rule{1.25cm}{0pt}}{\thechapter}{1em}{}						%начертание заголовка главы
\titleformat{\section}{\rule{1.25cm}{0pt}}{\thesection}{1em}{}						%начертание заголовка раздела
\titleformat{\subsection}{\rule{1.25cm}{0pt}}{\thesubsection}{1em}{}				%начертание заголовка подраздела
\titlespacing{\chapter}{0pt}{-10pt}{15pt}											%отступ после заголовка главы

\usepackage{tocloft}																%редактировать toc
\renewcommand{\cftchapfont}{}														%не выделять полужирным главы
\renewcommand{\cfttoctitlefont}{}													%размер названия toc
\renewcommand{\cftchappagefont}{}													%начертание номеров страниц
\renewcommand{\cftchapleader}{\cftdotfill\cftdotsep}				 				%заполнитель после названия главы
\setlength{\cftaftertoctitleskip}{5pt}												%отступ после заголовка оглавления
\setlength{\cftbeforetoctitleskip}{0pt}												%отступ до заголовка оглавления

\usepackage{multirow}																%объединение строк в таблицах

\bibliographystyle{gost780sCorr}														%стиль оформления литературы

\newtheorem{theorem}{Теорема}[chapter]												%нумерация теорем по главам
\newtheorem{corollary}{Следствие}[chapter]											%нумерация следствий по главам

\newcommand{\La}{\mathbf\Lambda}													%жирные буквы для обозначения векторов
\newcommand{\Q}{\mathbf{Q}}
\newcommand{\I}{\mathbf{I}}
\newcommand{\B}{\mathbf{B}}
\newcommand{\y}{\mathbf{y}}
\newcommand{\ff}{\mathbf{f}}
\newcommand{\R}{\mathbf{R}}
\newcommand{\g}{\mathbf{g}}
\newcommand{\p}{\mbox{\boldmath$\varphi$}}
\newcommand{\e}{\mathbf{e}}
\newcommand{\h}{\mathbf{H}}
\newcommand{\F}{\mathbf{F}}

\usepackage{tablestyle}																%подписи таблиц по левому краю